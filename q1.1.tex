\expandafter\ifx\csname ifdraft\endcsname\relax
 \documentclass{jsarticle}
 \begin{document}
\fi
\section  {chap1}
\subsection {question1.1}
(a) この問いでは、系は静的な平衡にあるという仮定がされている。
静的な平衡では自由電子がその場所に電場がない場合にのみその場所に存在することができる。
そうでなければ、自由電子は電場によって加速されるだろう。故に、導体内部の電場は0である。
導体の表面では、電荷は導体の表面と垂直な方向へ束縛されている。
そして導体の内部、外部で存在しうる電場は、導体表面に対して垂直な電場のみである。
\\
表面がガウス平面内にある任意の閉曲面においてガウスの法則を用いて、
\[
  \oint_S \mathbf{E} \cdot \mathbf{n} da = \frac{q}{\epsilon_0}
\]
ここで閉曲面の内部ではEは0なので、その表面積分も当然0である。よって、導体内部ではq=0であることがわかった。\\\\
(b) 前半部分の解答、導体内部に電場はなく中空部にも電荷はないので、ガウスの法則を用いて、
\[
  \oint_S \mathbf{E} \cdot \mathbf{n} da = 0
\]
従って、閉じた中空の導体の中空部は常に外部の電荷による電場から遮蔽される。
\\
 後半部分、導体を囲む閉曲面を考えてガウスの法則を用いると電荷qを得る。しかし、表面がガウス平面内にある任意の閉曲面においては、
\[
  \oint_S \mathbf{E} \cdot \mathbf{n} da = 0
\]
が満たされるので、導体は内部の表面に電荷を持ち、外部の表面にも電荷
を持つことがわかる。従って、中級部に置かれた電荷による電場からは導体の外部は遮蔽されない。
\\
コメント:前半部分は導体が接地されていても同じ結果を得る。
\\
\\
(c) 導体表面における電場が表面に垂直であることはすでに述べた。\\
表面状の面電荷密度が$\sigma$で与えられたとする。ここで、半分が導体中にもう半分が導体の外にあるような閉曲面を考えガウスの法則を用いる。なお、辺の長さがlの十分に小さな立方体を考え、2面が導体の表面に対して平行であるとする。
\[
  \oint_S \mathbf{E} \cdot \mathbf{n} da = \frac{ \sigma l^2 }{\epsilon_0}
\]
ここで、面積分は導体の表面に垂直な成分しか寄与せず、導体中の電場は0であるから、
\[
  \mathbf{E} l^2 = \frac{ \sigma l^2 }{\epsilon_0}
\]
よって\[
  \mathbf{E} = \frac{ \sigma}{\epsilon_0}
\]
\newpage


\expandafter\ifx\csname ifdraft\endcsname\relax
  \end{document}
\fi